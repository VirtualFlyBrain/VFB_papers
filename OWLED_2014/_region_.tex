\message{ !name(VFB_owled2014.tex)}
%%%%%%%%%%%%%%%%%%%%%%% file typeinst.tex %%%%%%%%%%%%%%%%%%%%%%%%%
%
% This is the LaTeX source for the instructions to authors using
% the LaTeX document class 'llncs.cls' for contributions to
% the Lecture Notes in Computer Sciences series.
% http://www.springer.com/lncs       Springer Heidelberg 2006/05/04
%
% It may be used as a template for your own input - copy it
% to a new file with a new name and use it as the basis
% for your article.
%
% NB: the document class 'llncs' has its own and detailed documentation, see
% ftp://ftp.springer.de/data/pubftp/pub/tex/latex/llncs/latex2e/llncsdoc.pdf
%
%%%%%%%%%%%%%%%%%%%%%%%%%%%%%%%%%%%%%%%%%%%%%%%%%%%%%%%%%%%%%%%%%%%


\documentclass[runningheads,a4paper]{llncs}

\usepackage{amssymb}
\setcounter{tocdepth}{3}
\usepackage{graphicx}
%\usepackage{hyphenat} see
%http://www.ctan.org/tex-archive/macros/latex/contrib/hyphenat - would
%be good to enable this to prevent hyphenation of OWL MS statements.
%Or find alternative

\usepackage{url}
\urldef{\mailsa}\path|{alfred.hofmann, ursula.barth, ingrid.haas, frank.holzwarth,|
\urldef{\mailsb}\path|anna.kramer, leonie.kunz, christine.reiss, nicole.sator,|
\urldef{\mailsc}\path|erika.siebert-cole, peter.strasser, lncs}@springer.com|    
\newcommand{\keywords}[1]{\par\addvspace\baselineskip
\noindent\keywordname\enspace\ignorespaces#1}

\begin{document}

\message{ !name(VFB_owled2014.tex) !offset(254) }


We treat expression patterns as anatomical structures defined as the
mereological sum of all cells that express a particular gene or
transgene. For all individual images of neurons, expression patterns
and neuron clones, we record what is being expressed, but individual
neurons and clones are typically fragments of expression patterns.
Users would find it very unintuitive if queries for expression
patterns mixed in images of expression pattern fragments.   But once
an expression pattern has been found, it is useful to get a list of
components.  


We define expresses using slightly more expressive logic than OWL can cope with:
\begin{quote} 
x \textbf{expresses} y iff: 
x \textbf{has\_part} some cell
For all cell(c) and \textbf{part\_of} y: (`gene expression' \textit{and} \textbf{has\_product} \textit{some} y) \textbf{occurs\_in} c
\end{quote}

An expression pattern of a gene/transgene can then be defined and its
partonomy populated using the pattern:

\begin{quote} 
gene B expression pattern
\textit{EquivalentTo}: `expression pattern' \textit{that}
\textbf{expresses} \textit{some} `gene B'
\textit{GCI}: \textbf{expresses} \textbf{some} gene B \textit{EquivalentTo} \textbf{part\_of} \textit{some} 'B expression pattern'
\end{quote}

 
Future versions of VFB are likely to use OWL versions of annotations, generated from the database, in the form of \textit{subClassOf} axioms asserting \textbf{overlaps} on gene expression classes.

e.g. 
\begin{quote} 
`expression pattern of gene B' \textit{subClassOf} \textbf{overlaps} \textit{some} `fan-shaped body'
\end{quote}



\message{ !name(VFB_owled2014.tex) !offset(580) }

\end{document}
